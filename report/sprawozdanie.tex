%%%%%%%%%%%%%%%%%%%%%%%%%%%%%%%%%%%%%%%%%
% fphw Assignment
% LaTeX Template
% Version 1.0 (27/04/2019)
%
% This template originates from:
% https://www.LaTeXTemplates.com
%
% Authors:
% Class by Felipe Portales-Oliva (f.portales.oliva@gmail.com) with template
% content and modifications by Vel (vel@LaTeXTemplates.com)
%
% Template (this file) License:
% CC BY-NC-SA 3.0 (http://creativecommons.org/licenses/by-nc-sa/3.0/)
%
%%%%%%%%%%%%%%%%%%%%%%%%%%%%%%%%%%%%%%%%%

%----------------------------------------------------------------------------------------
%	PACKAGES AND OTHER DOCUMENT CONFIGURATIONS
%----------------------------------------------------------------------------------------

\documentclass[
	12pt, % Default font size, values between 10pt-12pt are allowed
	%letterpaper, % Uncomment for US letter paper size
	%spanish, % Uncomment for Spanish
]{fphw}

% Template-specific packages
\usepackage[utf8]{inputenc} % Required for inputting international characters
\usepackage[T1]{fontenc} % Output font encoding for international characters
\usepackage{mathpazo} % Use the Palatino font

\usepackage{graphicx} % Required for including images

\usepackage{booktabs} % Required for better horizontal rules in tables

\usepackage{listings} % Required for insertion of code

\usepackage{enumerate} % To modify the enumerate environment
\usepackage{textcomp}
\usepackage{amsmath}
\usepackage{subcaption}
\usepackage[justification=centering]{caption}
\usepackage{float}
\usepackage{xcolor}
\usepackage{polski}
\usepackage[polish]{babel}
\usepackage{csvsimple}
\usepackage{mathtools}
\usepackage{ulem}

\definecolor{codegreen}{rgb}{0,0.6,0}
\definecolor{codegray}{rgb}{0.5,0.5,0.5}
\definecolor{codepurple}{rgb}{0.58,0,0.82}
\definecolor{backcolour}{rgb}{0.95,0.95,0.92}

\lstdefinestyle{mystyle}{
backgroundcolor=\color{backcolour},
commentstyle=\color{codegreen},
keywordstyle=\color{magenta},
numberstyle=\tiny\color{codegray},
stringstyle=\color{codepurple},
basicstyle=\ttfamily\footnotesize,
breakatwhitespace=false,
breaklines=true,
captionpos=b,
keepspaces=true,
numbers=left,
numbersep=5pt,
showspaces=false,
showstringspaces=false,
showtabs=false,
tabsize=2
}

\lstset{style=mystyle}

\renewcommand{\lstlistlistingname}{Spis listingów}
%----------------------------------------------------------------------------------------
%	ASSIGNMENT INFORMATION
%----------------------------------------------------------------------------------------

\title{Symulator chłodzenia pręta w oleju chłodzącym} % Assignment title

\author{Monika Nawój} % Student name

\date{14.11.2020} % Due date

\institute{Politechnika Warszawska \\ Wydział Elektryczny} % Institute or school name

\class{Metody Numeryczne 2020Z} % Course or class name

\professor{Robert Szmurło} % Professor or teacher in charge of the assignment

%----------------------------------------------------------------------------------------

\begin{document}

\maketitle % Output the assignment title, created automatically using the information in the custom commands above

%----------------------------------------------------------------------------------------
%	ASSIGNMENT CONTENT
%----------------------------------------------------------------------------------------

\section{Wstęp}
Celem projektu zaliczeniowego jest implementacja wybranych metod numerycznych
oraz ich weryfikacja na podstawie zdobytej wiedzy oraz prób eksperymentalnych.
Weryfikacja projektu zawiera określenie złożoności obliczeniowej oraz badanie
dokładności rozwiązania w zależności od parametrów.
Implementacja ninejszego projektu została zrealizowana w środowisku \textbf{MATLAB}.
\\
\\
Zasadniczym zagadnieniem projektu jest zbudowanie modelu numerycznego dla procesu chłodzenia
metalowego pręta w pojemniku wypełnionego cieczą (\textit{specjalnym oleju chłodzącym}).
Za pomocą tego modelu należy rozwiązać problem pewnego procesu technologicznego,
który obejmuje konieczność prędkiego schłodzenia sporej ilośći prętów w adekwatnym czasie.s
Zadanie to obejmuje również część optymalizacyjną, polegającą na wyznaczeniu
najkorzystniejszych parametrów dotyczących zbiorników z cieczą chłodzącą
oraz potrzebnych ich ilości w celu zagwarantowania ciągłej pracy.
\\
\\
Przykładem takiego procesu jest zadanie polegające na schłodzeniu 100000 prętów o masie \( m_b \) 0.25 kg
w czasie nieprzekraczającym 24 godzin oraz z optymalnymi kosztami chłodzenia.
Dodatkowymi parametrami w tym procesie to: czas wymiany zbiornika wynoszący 20 sekund,
koszt płynu chłodzącego (20 zł za litr), czas wymiany pręta (5 sekund) oraz koszt jednego zbiornika,
który jest związany z jego pojemnością i wynosi 100 zł za litr.
Cześcią tego zadania inżynierskiego jest określenie ilości pojemników potrzebnych
do działania całego procesu, przy założeniu że słodzenie jednego pojemnika ze 125 \textdegree{}C do 25 \textdegree{}C
trwa 4 godziny.
\\
\\
Obserwacja zmian temperatury pręta \( T_b \) oraz oleju \( T_w \) zostałą użyta do modelowania
fizycznego. Ciepło przenika przez ścianki pręta o powierzchni \( A \) ze stałym eksperymentalnie
określionym współczinnikiem \( A \), który w przeprowadzonych symulacjach przyjmuje wartość: \(0.0109 m^2\).
Materiał, z którego został zbudowany pręt charakteryzuje się pewną zdolnością kumulacji ciepła,
która jest odniesiona do masy pręta \( m_b \) oraz jest opisana za pomocą pojemności cieplnej \( c_p \).
Analogicznia charakterystyka odnosi się do oleju o masie: \( m_w \) oraz pojemności cieplnej: \( c_w \).

\newpage

W \textbf{modelu matematycznym} przyjmujemy założenie, że proces wymiany ciepła
jest opisany następującym układem równań różniczkowych zwyczajnych,
w którym obserwujemy dwie zmiany stanu: temperaturę pręta oraz cieczy:

\begin{flalign*}
	 &\frac{m_b c_b}{hA} \frac{dT_b}{dt} + T_b = T_w \\
	 &\frac{m_w c_w}{hA} \frac{dT_w}{dt} + T_w = T_b
\end{flalign*}

gdzie: \\
\( m_b \) - masa pręta \\
\( c_b \) - pojemność cieplna pręta \\
\( h \) - współczynnik przewodnictwa cieplnego \\
\( A \) -  powierzchnia pręta \\
\( m_w \) - masa płynu chłodzącego \\
\( c_w(T) \) - pojemność cieplna płynu chłodzącego zależna od temperatury T \\
\\
W modelu przyjmujemy założenie, że temperatura pręta oraz oleju jest stała w całej przestrzeni.
Założenie to zakłada, że mamy wysoką wartość współczynnika przewodnictwa cieplnego, równocześnie
zakładamy, że pomijamy to ograniczenie.

\section{Część 1}
\subsection{Wstęp}
Część pierwsza niniejszego projektu zakłada zaprojektowanie
oraz zaimplementowanie symulatora przebiegów czasowych temperatury pręta
oraz oleju chłodzącego.
Przyjęta została temperatura początkowa pręta \( T_b(0) = 1200 \)\textdegree{}C oraz początkową temperaturę
oleju chłodzącego \( T_w(0) = 25 \)\textdegree{}C.
Do znalezienia przybliżonego rozwiązania tego problemu zostanie użyta \textbf{metoda prosta Eulera}
oraz \textbf{metoda ulepszona Eulera}.

\subsubsection{Metoda prosta Eulera}
Zakładając, że istnieje jedyne rozwiązanie problemu początkowego (Cauchy'ego):
\begin{flalign*}
	& y' = f(x, y) \\
	& y(x_0) = y_0
\end{flalign*}
Możemy znaleźć przybliżone rozwiązanie metodą prostą Eulera. \\
\\
Rozwiązanie będzie szukane w przedziale \(<a,b>\), gdzie \( a = x_0\).
Przedział ten jest dzielony na \(n\) części o długości \(h\). \\
Wartości funkcji będącej rozwiązaniem danego zagadnienia będziemy liczyć ze wzoru:
\begin{flalign*}
	y_n = y_{n-1} + hf(x_{n-1}, y_{n-1})
\end{flalign*}

\subsubsection{Metoda ulepszona Eulera}
Zakładając, że istnieje jedyne rozwiązanie problemu początkowego (Cauchy'ego):
\begin{flalign*}
	& y' = f(x, y) \\
	& y(x_0) = y_0
\end{flalign*}
Możemy znaleźć przybliżone rozwiązanie metodą ulepszoną Eulera. \\
\\
Rozwiązanie będzie szukane w przedziale \(<a,b>\), gdzie \( a = x_0\).
Przedział ten jest dzielony na \(n\) części o długości \(h\). \\
Punkty podziału: \\
\(x_n = a + ih \), gdzie \(i = 0,1, \dots, n\). \\
Wartości funkcji będącej rozwiązaniem danego zagadnienia będziemy liczyć ze wzoru:
\begin{flalign*}
	y_n = y_{n-1} + hf(x_{n-1} + \frac{h}{2}, y_{n-1} + \frac{h}{2}f(x_{n-1}, y_{n-1}))
\end{flalign*}

\subsubsection{Błąd względny}
Dużymi literami \(A, B, C, \dots\) oznaczone są liczby dokładne,
a małymi \(a, b, b, \dots\) przybliżone wartości tych liczb.

\textbf{Błąd bezwzględny} liczby przybliżonej stanowi wartość bezwględną różnicy pomiędzy liczbą dokładną,
a jej przybliżeniem.
\begin{flalign*}
	\Delta a = |A - a|
\end{flalign*}

\textbf{Błąd względny} stanowi stosunek błędu bezwględnego do wartości liczby przybliżonej dla wartości przybliżonej różnej od zera..
\begin{flalign*}
	\delta a = \frac{\delta a}{| a |}
\end{flalign*}

\subsection{Podsumowanie współczynników}
Przebiegi czasowe temperatury pręta oraz temperatury oleju chłodzącego zostały przeprowadzone dla poniższych współczynników: \\
\begin{flalign*}
	& h = 160 \quad [\frac{J}{s \cdot m^2}] \\
	& A = 0.0109 \quad [m^2] \\
	& m_b = 0.2 \quad [kg] \\
	& m_w = 2.5 \quad [kg] \\
	& c_b = 3.85 \quad [\frac{J}{kg \cdot K}] \\
	& cw = 4.1813 \quad [\frac{J}{kg \cdot K}] \\
\end{flalign*}

\subsection{Kod źródłowy programu}
Ze względu na reużywalność niektórych funkcji programu w dalszych etapach projektu,
kod źródłowy dla części pierwszej został podzielony na 5 plików: \\
\lstinputlisting[language=MatLab, caption=Funkcja sterująca z danymi parametrami oraz rysująca wykresy]{../src/part1.m}

\lstinputlisting[language=MatLab, caption=Metoda prosta Eulera]{../src/myEuler.m}

\newpage

\lstinputlisting[language=MatLab, caption=Ulepszona metoda Eulera]{../src/improvedEuler.m}

\lstinputlisting[language=MatLab, caption=Funkcja wykonująca całkowanie zadanego równania stanu]{../src/tempModel.m}

\lstinputlisting[language=MatLab, caption=Funkcja licząca błąd względny]{../src/relativeError.m}

\newpage

\subsection{Interpretacja wyników}

\subsubsection{Krok całkowania h}
Krok całkowania \(h\) został dobrany metodą eksperymentalną z przedziału: 0.001 - 1.

\begin{figure}[H]
	\includegraphics[width=\linewidth]{../assets/part1/krok-h-1.png}
	\caption{Parametry: krok całkowania(0) \(h = 0.001\), \(T_b = 1200\)(0) [\textdegree{}C], \(T_w = 25\)[\textdegree{}C], \(m_w = 2.5\)[kg], \(t = 3\)[s]}
	\label{fig:krok-1}
\end{figure}

\begin{figure}[H]
	\includegraphics[width=\linewidth]{../assets/part1/krok-h-2.png}
	\caption{Parametry: krok całkowania(0) \(h = 0.01\), \(T_b = 1200\)(0) [\textdegree{}C], \(T_w = 25\)[\textdegree{}C], \(m_w = 2.5\)[kg], \(t = 3\)[s]}
	\label{fig:krok-2}

	\includegraphics[width=\linewidth]{../assets/part1/krok-h-3.png}
	\caption{Parametry: krok całkowania(0) \(h = 0.1\), \(T_b = 1200\)(0) [\textdegree{}C], \(T_w = 25\)[\textdegree{}C], \(m_w = 2.5\)[kg], \(t = 3\)[s]}
	\label{fig:krok-3}
\end{figure}

\begin{figure}[H]
	\includegraphics[width=\linewidth]{../assets/part1/krok-h-4.png}
	\caption{Parametry: krok całkowania(0) \(h = 1\), \(T_b = 1200\) [\(0\)textdegree{}C], \(T_w = 25\)[\textdegree{}C], \(m_w = 2.5\)[kg], \(t = 3\)[s]}
	\label{fig:krok-4}
\end{figure}

Dla bardzo małego kroku całkowania \(h\) \textit{nie widać} wyraźnej różnicy między przebiegiem temperatury
obliczonym za pomocą \textbf{metody Eulera}, a tym obliczonym za pomocą \textbf{ulepszonej metody Eulera}.
Jest to widoczne na rysunkach \ref{fig:krok-1} oraz \ref{fig:krok-2}.\\
Na rysunku \ref{fig:krok-3} widać wyraźną różnicę w przebiegu temperatury pręta przy kroku całkowania \(h = 0.1\).
Wykres przebiegu temperatury obliczony za pomocą \textbf{ulepszonej metody Eulera} jest bardziej zbliżony,
do tego, do którego został użyty mniejszy krok całowania w poprzednich próbach eksperymentalnych. \\
Natomiast krok całkowania \(h = 1\) przedstawiony na rysunku \ref{fig:krok-4} okazał się zbyt duży,
aby przebieg temperatury obliczony przez dowolną z tych dwóch funkcji prezentował sensowne dane. \\
Za pomocą tego eksperymentu został wybrany krok całkowania \(h = 0.1\) i właśnie taki będzie używany
w następnych próbach eksperymentalnych. \\
Warto również zauważyć, że \textbf{ulepszona metoda Eulera} uzyskuje wyniki dokładniejsze
przy większym kroku całkowania niż metoda prosta Eulera.

\newpage

\subsubsection{Przebieg czasowy temperatury pręta oraz oleju chłodzącego}

W celu weryfikacji użyte zostały dane uzyskane z eksperymentów technicznych
fizycznych, w których obserwowano tempraturę pręta oraz temperaturę oleju chłodzącego
dla różnych ilości płynu po upływie określonego czasu \(t\) (2,3,4 lub 5 sekund).

\begin{table}[H]
	\begin{tabular}{ |c|c|c|c|c|c|c| }
		\hline
		 Nr pomiaru  & \( T_b(0) \) [\textdegree{}C] &  \( T_w(0) \) [\textdegree{}C] & \(M_w\) [kg] & \(t\)[s] &  \( T_b(t) \) [\textdegree{}C]  & \( T_w(t) \) [\textdegree{}C] \\
		\hline
		 1  & 1200 & 25 & 2.5 & 3 & 107.7 & 105.1  \\
		\hline
		 2  & 800 & 25 & 2.5 & 3 & 79.1 & 78.0  \\
		\hline
		 3  & 1100 & 70 & 2.5 & 3 & 142.1 & 139.1  \\
		\hline
		 4  & 1200 & 25 & 2.5 & 5 & 105.7 & 105.5  \\
		\hline
		 5  & 800 & 25 & 2.5 & 5 & 78.2 & 78.1  \\
		\hline
		 6  & 1100 & 70 & 2.5 & 2 & 150.1 & 138.2  \\
		\hline
		 7  & 1100 & 70 & 5 & 2 & 116.6 & 105.1  \\
		\hline
		 8  & 1100 & 70 & 10 & 2 & 99.1 & 88.1  \\
		\hline
		 9  & 1100 & 70 & 2.5 & 4 & 141.2 & 139.8  \\
		\hline
		 10  & 1100 & 70 & 2.5 & 5 & 140.9 & 140.1  \\
		\hline
	\end{tabular}
	\caption{Pomiary eksperymentowe.}
\end{table}


\begin{figure}[H]
	\includegraphics[width=\linewidth]{../assets/part1/przebieg-czasowy-1.png}
	\caption{Parametry: \(T_b(0) = 1200\) [\textdegree{}C], \(T_w(0) = 25\)[\textdegree{}C], \(m_w = 2.5\)[kg], \(t = 3\)[s]}

	\includegraphics[width=\linewidth]{../assets/part1/przebieg-czasowy-2.png}
	\caption{Parametry: \(T_b(0) = 800\) [\textdegree{}C], \(T_w(0) = 25\)[\textdegree{}C], \(m_w = 2.5\)[kg], \(t = 3\)[s]}
\end{figure}


\begin{figure}[H]
	\includegraphics[width=\linewidth]{../assets/part1/przebieg-czasowy-3.png}
	\caption{Parametry: \(T_b(0) = 1100\) [\textdegree{}C], \(T_w(0) = 70\)[\textdegree{}C], \(m_w = 2.5\)[kg], \(t = 3\)[s]}

	\includegraphics[width=\linewidth]{../assets/part1/przebieg-czasowy-4.png}
	\caption{Parametry: \(T_b(0) = 1200\) [\textdegree{}C], \(T_w(0) = 25\)[\textdegree{}C], \(m_w = 2.5\)[kg], \(t = 5\)[s]}
\end{figure}

\begin{figure}[H]
	\includegraphics[width=\linewidth]{../assets/part1/przebieg-czasowy-5.png}
	\caption{Parametry: \(T_b(0) = 800\) [\textdegree{}C], \(T_w(0) = 25\)[\textdegree{}C], \(m_w = 2.5\)[kg], \(t = 5\)[s]}

	\includegraphics[width=\linewidth]{../assets/part1/przebieg-czasowy-6.png}
	\caption{Parametry: \(T_b(0) = 1100\) [\textdegree{}C], \(T_w(0) = 70\)[\textdegree{}C], \(m_w = 2.5\)[kg], \(t = 2\)[s]}
\end{figure}

\begin{figure}[H]
	\includegraphics[width=\linewidth]{../assets/part1/przebieg-czasowy-7.png}
	\caption{Parametry: \(T_b(0) = 1100\) [\textdegree{}C], \(T_w(0) = 70\)[\textdegree{}C], \(m_w = 5\)[kg], \(t = 2\)[s]}

	\includegraphics[width=\linewidth]{../assets/part1/przebieg-czasowy-8.png}
	\caption{Parametry: \(T_b(0) = 1100\) [\textdegree{}C], \(T_w(0) = 70\)[\textdegree{}C], \(m_w = 10\)[kg], \(t = 2\)[s]}
\end{figure}

\begin{figure}[H]
	\includegraphics[width=\linewidth]{../assets/part1/przebieg-czasowy-9.png}
	\caption{Parametry: \(T_b(0) = 1100\) [\textdegree{}C], \(T_w(0) = 70\)[\textdegree{}C], \(m_w = 2.5\)[kg], \(t = 4\)[s]}

	\includegraphics[width=\linewidth]{../assets/part1/przebieg-czasowy-10.png}
	\caption{Parametry: \(T_b(0) = 1100\) [\textdegree{}C], \(T_w(0) = 70\)[\textdegree{}C], \(m_w = 2.5\)[kg], \(t = 5\)[s]}
\end{figure}

Wyniki przedstawionych powyżej prób eksperymentalnych zostały przedstawione w tabeli podsumowującej
wraz z wynikiem oczekiwanym z pomiarów eksperymentów technicznych fizycznym oraz obliczonym dla tej wartości
błędem względnym.

\begin{table}[H]
	\begin{tabular}{|c|c|c|c|}\hline%
		Nr próby & oczekiwane \(T_b(t)\)[\textdegree{}C] & przybliżone \(T_b(t)\) [\textdegree{}C] & \(\delta T_b(t)\) \\\hline
		\csvreader[no head, late after line=\\\hline]%
		{../assets/part1/euler-wyniki.csv}{}%
		{\thecsvrow&\csvcoli&\csvcoliii&\csvcoliv}%
	\end{tabular}
	\caption{Temperatury końcowe pręta przybliżone za pomocą metody prostej Eulera}
	\label{tab:euler-1}
\end{table}

\begin{table}[H]
	\begin{tabular}{|c|c|c|c|}\hline%
		Nr próby & oczekiwane \(T_w(t)\)[\textdegree{}C] & przybliżone \(T_w(t)\) [\textdegree{}C] & \(\delta T_w(t)\) \\\hline
		\csvreader[no head, late after line=\\\hline]%
		{../assets/part1/euler-wyniki.csv}{}%
		{\thecsvrow&\csvcolii&\csvcolv&\csvcolvi}%
	\end{tabular}
	\caption{Temperatury końcowe oleju chłodzącego przybliżone za pomocą metody prostej Eulera}
	\label{tab:euler-2}
\end{table}

\begin{table}[H]
	\begin{tabular}{|c|c|c|c|}\hline%
		Nr próby & oczekiwane \(T_b(t)\)[\textdegree{}C] & przybliżone \(T_b(t)\) [\textdegree{}C] & \(\delta T_b(t)\) \\\hline
		\csvreader[no head, late after line=\\\hline]%
		{../assets/part1/ulepszony-euler-wyniki.csv}{}%
		{\thecsvrow&\csvcoli&\csvcoliii&\csvcoliv}%
	\end{tabular}
	\caption{Temperatury końcowe pręta przybliżone za pomocą ulepszonej metody Eulera}
	\label{tab:impoved-euler-1}
\end{table}

\begin{table}[H]
	\begin{tabular}{|c|c|c|c|}\hline%
		Nr próby & oczekiwane \(T_w(t)\)[\textdegree{}C] & przybliżone \(T_w(t)\) [\textdegree{}C] & \(\delta T_w(t)\) \\\hline
		\csvreader[no head, late after line=\\\hline]%
		{../assets/part1/ulepszony-euler-wyniki.csv}{}%
		{\thecsvrow&\csvcolii&\csvcolv&\csvcolvi}%
	\end{tabular}
	\caption{Temperatury końcowe oleju chłodzącego przybliżone za pomocą ulepszonej metody Eulera}
	\label{tab:impoved-euler-2}
\end{table}

Przedstawione w tabelach powyżej wartości błędów bezwględnych wynoszą wartość bliską zera,
co wskazuje na poprawność symulacji przebiegu czasowego temperatury pręta oraz oleju chłodzącego. \\
Błędy te mogą wynikać z kilku przyczyn. \\
Jedną z nich jest wykorzystanie danych przybliżonych do 4 miejsca po przecinku w przypadku przeprowadzoncyh
prób, oraz danych przybliżonych do 1 miejsca po przecinku w przypadku danych z eksperymentu fizycznego.
Dodatkowo możemy wziąć pod uwagę nieścisłości związane z dokładnością pomiarów eksperymentu fizycznego.
Oczywiście musimy również brać pod uwagę dokładność algorytmu, który został wykorzystany do symulacji. \\
\\
Błąd względny dla temperatury oleju chłodzącego \(T_w(t)\) nie różni się znacznie dla wartości przybliżonych
\textit{metodą prostą Eulera} (Tabela \ref{tab:euler-2}), a tych obliczonych \textit{ulepszoną metodą Eulera}
(Tabela \ref{tab:impoved-euler-2}). \\
Natomiast patrząc na różnice wartości błędu względnego dla temperatury pręta \(T_b(t)\) widać,
że błąd względny wartości przybliżonej \textbf{ulepszoną metodą Eulera} (Tabela \ref{tab:impoved-euler-1}) jest znacznie mniejszy
od błędu względnego wartości przybliżonej \textbf{metodą prostą Eulera} (Tabela \ref{tab:euler-1}).
\subsection{Wnioski}

Zaprezentowane poniżej eksperymenty dowiodły, iż symulacja przebiegu czasowego temperatury pręta
oraz oleju chłodzącego przebiega poprawnie. \\
Dodatkowo \textbf{ulepszona metoda Eulera} przy większym kroku całkowania przybliża wartości
z mniejszym błędem względnym niż \textbf{metoda prosta Eulera}.
Z tego względu metoda ulepszona będzie stosowana w następnych częściach projektu.

\newpage

\section{Część 2}
\subsection{Wstęp}
W części drugiej przyjmujemy, że współczynnik przewodnictwa cieplnego \(h\) między prętem a olejem chłodzącym
jest nieliniowy i zależny od różnicy temperatur.
Współczynnik ten jest opisany poniższymi zależnościami oraz tabelą z danymi pomiarowymi.
\begin{flalign*}
	h = f_h(\Delta T) = f_h(T_b - Tw)
\end{flalign*}

\begin{table}[H]
	\begin{tabular}{ |c|c|c|c|c|c|c|c|c|c|c|c|c| }
		\hline
		\( \Delta T \) [\textdegree{}C] &  -1500 & -1000 & -300 & -50 & -1 & 1 & 20 & 50 & 200 & 400 & 1000 & 2000 \\
		\hline
		\(h[W \cdot m^{-2}]\) & 178 & 176 & 168 & 161 & 160 & 160 & 160.2 & 161 & 165 & 168 & 174 & 179 \\
		\hline
	\end{tabular}
	\caption{Dane pomiarowe współczynnika przewodnictwa.}
\end{table}

Dane te będą aproksymowane oraz interpolowane, aby można było je wykorzystać w symulacji. \\
Przyjęta została temperatura początkowa pręta \(T_b(0)=1200\)\textdegree{}C,
a temperatura początkowa oleju chłodzącego \(T_w(0)=25\)\textdegree{}C.
\subsubsection{Podzadanie 1}
W zakresie podzadania pierwszego należało zaimplementować dwa algorytmy pozwalające znaleźć przybliżenie
współczynnika przewodnictwa cieplnego \(h\) na podstawie danych z tabeli,
a następnie wykorzystać te przybliżenia w symulacji. \\
Zostaną zaimplementowane następujące algorytmy:
\begin{itemize}
\item algorytm aproksymacji wielomianowej metodą najmniejszych kwadratów
\item algorytm interpolacji wykorzystujący funkcje sklejane trzeciego stopnia
\end{itemize}

\subsubsection{Podzadanie 2}
W podzadaniu drugim należy porównać przebiegi krzywej reprezentującej współczynnik
przewodnictwa cieplnego uzyskanej z obu metod z poprzedniego podzadania (aproksymacji oraz interpolacji).
Różnicę tą należy przedstawić na jednym wykresie. \\
Dodatkowo należy obliczyć średnią różnicę przebiegów charakterystyk współczynników różnicy temperatur
\(\Delta T\) odpowiadającym zakresowi z tabeli za pomocą poniższego wzoru:
\begin{flalign*}
	e_{avg} = \frac{1}{3500} \int_{-1500}^{2000} |h_{aproksymowane} - h_{interpolowane}|d(\Delta T)
\end{flalign*}
Jest to całka wartości bezwzględnej różnicy współczynników wykonana względem \(\Delta T\)
w przedziale -1500 do 2000.
Całka ta zostanie wyznaczona za pomocą metody parabol.

\subsubsection{Podzadanie 3}
Podzadanie 3 polega na sprawdzeniu wpływu modelu aproksymowanego charakterystykę współczynnika \(h\)
na wyniki symulacji rozkładu temperatury pręta oraz oleju chłodzącego.

\subsubsection{Aproksymacja wielomianowa metodą najmniejszych kwadratów}
\textbf{Aproksymacja} służy do znajdowania przybliżonych wartości funkcji \(f(x)\) w dowolnym
punkcie przedziału \(<a, b>\).
Funkcja aproksymacyjna różni się od funkcji interpolacyjnej tym, że nie wymagamy od funkcji aproksymacyjnej
aby pokrywała się w pewnych punktach z funkcją aproksymowaną.
Jedynym wymaganiem jest aby funkcja aproksymacyjna była \textit{bliska} funkcji aproksymowanej. \\
Załóżmy, że z doświadczeń lub pomiarów określiliśmy w n+1 różnych punktach:
\begin{flalign*}
	x_0,x_1,x_2, \dots, x_n
\end{flalign*}
z przedziału \(<a, b>\) wartości funkcji \(y = f(x)\) i te wartości oznaczamy przez:
\begin{flalign*}
	y_0 = f(x_0), y_1 = f(x_1), y_2 = f(x_2), \dots, y_n = f(x_n),
\end{flalign*}
Rozpatrywane będą funkcje  aproksymujące w postaci wielomianów algebraicznych:
\begin{flalign*}
	W_m(x) = a_0 + a_1 x + a_2 x^2 + \dots + a_{m-1} x^{m-1} + a_m x^m
\end{flalign*}
gdzie \(a_i\) dla \(i=0,1,2, \dots, m\) to współczynniki rzeczywiste wielomianu, które trzeba znaleźć. \\
Bazą takiego wielomianu są funkcje: \(\{1,x,x^2, \dots, x^m\}\). \\
Budujemy macierz \(M\) dla tej bazy:
\begin{flalign*}
	M =
	\begin{bmatrix*}
		1 & x_0 & \dots & x_0^m \\
		1 & x_1 & \dots & x_1^m \\
		1 & \dots & \dots & \dots \\
		1 & x_n & \dots & x_n^m
		\end{bmatrix*}
\end{flalign*}
Aby znaleźć współczynniki \(a_i\) dla \(i=0,1,2, \dots, m\), z powyższą macierzą należy rozwiązać układ:
\begin{flalign*}
	M^T M \cdot A  = M^T \cdot Y
\end{flalign*}
gdzie:
\begin{flalign*}
	Y =
	\begin{bmatrix*}
		y_1 \\
		y_2 \\
		\dots \\
		y_n
	\end{bmatrix*}
	A =
	\begin{bmatrix*}
		a_0 \\
		a_1 \\
		\dots \\
		a_m
	\end{bmatrix*}
\end{flalign*}

\subsubsection{Błąd aproksymacji}
Dążymy do tego, aby funkcja aproksymująca była jak najbliższa funkcji aproksymowanej,
a więc aby \textit{błąd aproksymacji} był jak najmniejszy. \\
Do obliczenia błędu aproksymacji zastosujemy wzór na średni błąd przypadający na jeden węzeł:
\begin{flalign*}
	bl = \sqrt[]{\frac{\sum_{i=0}^{n}(y_i - F(x_i))^2 }{n+1}}
\end{flalign*}
Pod pierwiastkiem w liczniku znajduje się suma kwadratów odchyleń, którą minimalizowaliśmy,
natomiast w mianowniku jest ilość węzłów.

\newpage

\subsubsection{Interpolacja wykorzystująca funkcje sklejane trzeciego stopnia}
\textbf{Interpolacja} służy do znajdowania przybliżonych wartości funkcji \(f(x)\)
w dowolnym punkcie przedziału \(<a,b>\), nawet w przypadku gdy znane jest tylko
kilka wartości funkcji \(f(x)\) w tym przedziale. \\
Załóżmy, że z doświadczeń lub pomiarów określiliśmy w n+1 różnych punktach:
\begin{flalign*}
	x_0,x_1,x_2, \dots, x_n
\end{flalign*}
z przedziału \(<a, b>\) wartości funkcji \(y = f(x)\) i te wartości oznaczamy przez:
\begin{flalign*}
	y_0 = f(x_0), y_1 = f(x_1), y_2 = f(x_2), \dots, y_n = f(x_n),
\end{flalign*}
Zadaniem interpolacji jest wyznaczenie \textbf{funkcji interpolacyjnej} \(F(x)\),
określonej w przedziale \(<a,b>\), która w \textbf{węzłach interpolacji} \(x_0,x_1,x_2,\dots,x_n\)
przyjmuje wartości \textit{funkcji interpolowanej} \(f(x)\), a w punktach poza węzłami
przybliża wartość tej funkcji. \\
Zatem dla punktów \(x_i\) dla \(i = 0,1,2,\dots, n\) w przedziale \(<a,b>\) funkcja interpolacyjna
\(F(x)\) musi spełniać \(n+1\) warunków:
\begin{flalign*}
	F(x_i) = y_i = f(x_i)
\end{flalign*}
dla \(i=0,1,2,\dots, n\) \\
\\
Stosując do interpolacji wielomian interpolacyjny nie możemy narzucać stopnia wielomianu,
ponieważ ten zależy od ilości węzłów. \\
Natomiast stopień funkcji sklejanej (tzw.\textit{splajnu}) nie będzie zależał od ilości węzłów. \\
Rozpatrujemy przedział \(<a,b>\) i dzielimy go na \(n\) części,
czyli \(n\) podprzedziałów o długości:
\begin{flalign*}
	h = \frac{b-a}{n}
\end{flalign*}
Otrzymamy wtedy węzły równoległe: \(x_i = a + i * h\) dla \(i = 0,1, \dots, n\). \\
\\
\textbf{Funkcja sklejana 3-iego stopnia} jest to funkcja, która na każdym podprzedziale
jest \textit{wielomianem 3-iego stopnia} "posklejana" w taki sposób, aby była ciągła
i miała pierwszą i drugą pochodną ciągłą na przedziale \(<a,b>\). \\
\\
Aby precyzyjnie określić funkcję sklejaną 3-iego stopnia \(S_3(x)\) na przedziale \(<a,b>\)
należy na początku określić bazę splajnów 3-iego stopnia dla węzłów równoodległych. \\
Jedna funkcja bazowa jest podana za pomocą wzoru:
\begin{flalign*}
	\Phi_i(x) = \frac{1}{h^3}
	\begin{cases}
		(x-x_{i-2})^3 & \quad \text{dla } x \in <x_{i-2}, x_{i-1}>\\
		(x-x_{i-2})^3 - 4(x - x_{x_i-1})^3 & \quad \text{dla } x \in <x_{i-1}, x_{i}>\\
		(x_{i+2} - x)^3 - 4(x_{x_i+1} - x)^3 & \quad \text{dla } x \in <x_{i}, x_{i + 1}>\\
		(x_{i+2} - x)^3 & \quad \text{dla } x \in <x_{i+1}, x_{i + 2}>\\
		0 & \quad \text{dla } x \in R - <x_{i-2}, x_{i + 2}>\\
	\end{cases}
\end{flalign*}

\newpage

Powyższe wymagania muszą zostać spełnione tzn.:
\begin{itemize}
\item musi być wielomianem 3-iego stopnia na każdym podprzedziale
\item mieć pierwszą i drugą pochodną ciągłą na przedziale \(<a,b>\)
\item i-ta funkcja bazowa oznaczona jako \(\Phi_i(x) \) ma w i-tym węźle pochodną ciągłą na przedziale \(<a, b>\)
oraz \textit{maksimum} równe 4.
\item dodatkowo w węzłach obok ma wartość 1, a w węzłach o numerach \(i-2\) i \(i+2\) ma wartość 0
\end{itemize}

Po zastosowaniu funkcji sklejanej \(S_3(x)\) do interpolacji funkcji \(f(x)\) w przedziale \(<a,b>\).
otrzymujemy ze związków \(n+1\) równań, dla współczynników \(n+3\):
\begin{flalign*}
	S_3(x) = c_{-1} \Phi_{-1}(x) + c_0 \Phi_0(x) + c_1 \Phi_1(x) + \dots + c_n \Phi_n(x) + c_{n+1} \Phi_{n+1}(x)
\end{flalign*}
Nasz układ ma zatem dwa stopnie swobody i aby jednoznacznie wyznaczyć \(S_3'(x)\)
w punktach \(a\) oraz \(b\) należy zadać współczynniki kierunkowe stycznych
pod jakimi funkcja interpolacyjna ma startować z punktu \(a\) w prawo i jak ma
przechodzić do punktu \(b\) z lewej strony. \\
Dodatkowe warunki:
\begin{flalign*}
	S_3'(a^+) = \alpha, S_3'(b^-) = \beta
\end{flalign*}

Z warunków oraz własności funkcji bazowych i ich pochodnych otrzymujemy następujący układ równań:
\begin{flalign*}
	& -c_{-1} + c_1 = \frac{h}{3} \alpha \\
	& -c_{n-1} + c_{n+1} = \frac{h}{3} \beta
\end{flalign*}

Po rozwiązaniu powyższego układu równań i wyliczeniu współczynników \(c_{c-1}, c_{n+1}\),
wstawiamy je do następującego wzoru:
\begin{flalign*}
	& c_{i-1} + 4c_i + c_{i+1} = y_i \quad i=0,1,\dots,n
\end{flalign*}
Otrzymujemy wówczas następujący układ \(n+1\) równań liniowych z \(n+1\) niewiadomymi
\(c_0, c_1, \dots, c_n\):
\begin{flalign*}
	& 4c_0 + 2c_1 = y_0 + \frac{h}{3} \alpha \\
	& c_0 + 4c_1 + c_2 = y_1 \\
	& c_1 + 4c_2 + c_3 = y_2 \\
	&  \dots \\
	& c_{n-2} + 4c_{n-1} + c_n = y_{n-1} \\
	& 2c_{n-1} + 4c_n = y_n - \frac{h}{3} \beta
\end{flalign*}

Układ ten ma zawsze \textbf{jedno} rozwiązanie na \(c_0, c_1, \dots, c_n\), \\
pozostałe 2 współczynniki obliczamy z następujących wzorów:
\begin{flalign*}
	& c_{-1} = c_1 - \frac{h}{3} \alpha \\
	& c_{n+1} = c_{n-1} + \frac{h}{3} \beta
\end{flalign*}

\subsubsection{Wzór złożony parabol (Simpsona)}
Przedział \(<a,b>\) dzielimy na \(m\) części (\(m\) musi być \textbf{parzyste}):
\begin{flalign*}
	h = \frac{b-a}{m}, x_k = a + kh, \quad \text{gdzie } k=0,1,\dots,m
\end{flalign*}
Otrzymujemy \(m\) podprzedziałów o długości \(h\), inaczej \(m/2\)
podprzedziałów o długości \(2h\), w każdym z nich:
\([x_i, x_{i+2}]\), gdzie \(i=0,1,\dots,m-2\) stosujemy wzór prosty parabol:
\begin{flalign*}
	\int_{x_i}^{x_{i+2}} f(x)dx \cong \frac{h}{3} (f(x_i) + 4f(x_{i+1}) + f(x_{i+2}))
\end{flalign*}
\subsection{Podsumowanie współczynników}
\begin{flalign*}
	& h \text{ aproksymowane oraz interpolowane (ruchome) } \quad [\frac{J}{s \cdot m^2}] \\
	& A = 0.0109 \quad [m^2] \\
	& m_b = 0.2 \quad [kg] \\
	& m_w = 2.5 \quad [kg] \\
	& c_b = 3.85 \quad [\frac{J}{kg \cdot K}] \\
	& cw = 4.1813 \quad [\frac{J}{kg \cdot K}] \\
	& T_b(0) = 1200 \quad \text{[\textdegree{}C]} \\
	& T_w(0) = 25 \quad \text{[\textdegree{}C]} \\
\end{flalign*}

\newpage

\subsection{Kod źródłowy programu}
\subsubsection{Podzadanie 1}
\lstinputlisting[language=MatLab, caption=Funkcja sterująca z danymi parametrami oraz rysująca wykresy]{../src/part2_1.m}

\lstinputlisting[language=MatLab, caption=Aproksymacja wielomianowa metodą najmniejszych kwadratów]{../src/approx.m}

\lstinputlisting[language=MatLab, caption=Błąd aproksymacji]{../src/approxError.m}

\lstinputlisting[language=MatLab, caption=Interpolacja funkcjami sklejanymi 3-iego stopnia]{../src/spline.m}

\subsubsection{Podzadanie 2}
\lstinputlisting[language=MatLab, caption=Funkcja sterująca z danymi parametrami oraz rysująca wykresy]{../src/part2_2.m}

\lstinputlisting[language=MatLab, caption=Funkcja licząca średnią różnicę przebiegów charakterystyk współczynników różnicy temperatur]{../src/avgFactorDiff.m}

\lstinputlisting[language=MatLab, caption=Metoda złożona parabol]{../src/simpsonIntegral.m}


\subsubsection{Podzadanie 3}
\lstinputlisting[language=MatLab, caption=Funkcja sterująca z danymi parametrami oraz rysująca wykresy]{../src/part2_3.m}

\newpage

\subsection{Interpretacja wyników}
\subsubsection{Podzadanie 1}
Stopień wielomianu aproksymacji został dobrany metodą eksperymentalną z zakresu: 1-5.

\begin{table}[H]
	\centering
	\begin{tabular}{|c|c|}\hline%
	Stopień wielomianu & Błąd aproksymacji \\\hline
	\csvreader[no head, late after line=\\\hline]%
	{../assets/part2/approx-error.csv}{}%
	{\thecsvrow&\csvcoli}%
	\end{tabular}
	\caption{Stopień wielomianu oraz związany z nim błąd aproksymacji}
	\label{tab:approx-error}
\end{table}

\begin{figure}[H]
	\includegraphics[width=\linewidth]{../assets/part2/przewodnictwo-cieplne-h.png}
	\caption{Parametry: Stopień wielomianu aproksymacji: 4}
	\label{fig:przewodnictwo-h}
\end{figure}

\subsubsection{Podzadanie 2}
\begin{figure}[H]
	\includegraphics[width=\linewidth]{../assets/part2/roznica-h.png}
	\caption{Parametry: Stopień wielomianu aproksymacji: 4}
	\label{fig:roznica-h}
\end{figure}

Średnia różnica przebiegów charakterystyk współczynników
różnicy temperatur \(\Delta T\) wynosi:
\csvreader[no head]%
{../assets/part2/eavg.csv}{}%
{\csvcoli}%

\newpage

\subsubsection{Podzadanie 3}



\begin{figure}[H]
	\includegraphics[width=\linewidth]{../assets/part2/zmienny-h-1.png}
	\caption{Parametry: \(T_b(0) = 1200\) [\textdegree{}C], \(T_w(0) = 25\)[\textdegree{}C], \(m_w = 2.5\)[kg], \(t = 3\)[s]}

	\includegraphics[width=\linewidth]{../assets/part2/zmienny-h-2.png}
	\caption{Parametry: \(T_b(0) = 800\) [\textdegree{}C], \(T_w(0) = 25\)[\textdegree{}C], \(m_w = 2.5\)[kg], \(t = 3\)[s]}
\end{figure}


\begin{figure}[H]
	\includegraphics[width=\linewidth]{../assets/part2/zmienny-h-3.png}
	\caption{Parametry: \(T_b(0) = 1100\) [\textdegree{}C], \(T_w(0) = 70\)[\textdegree{}C], \(m_w = 2.5\)[kg], \(t = 3\)[s]}

	\includegraphics[width=\linewidth]{../assets/part2/zmienny-h-4.png}
	\caption{Parametry: \(T_b(0) = 1200\) [\textdegree{}C], \(T_w(0) = 25\)[\textdegree{}C], \(m_w = 2.5\)[kg], \(t = 5\)[s]}
\end{figure}

\begin{figure}[H]
	\includegraphics[width=\linewidth]{../assets/part2/zmienny-h-5.png}
	\caption{Parametry: \(T_b(0) = 800\) [\textdegree{}C], \(T_w(0) = 25\)[\textdegree{}C], \(m_w = 2.5\)[kg], \(t = 5\)[s]}

	\includegraphics[width=\linewidth]{../assets/part2/zmienny-h-6.png}
	\caption{Parametry: \(T_b(0) = 1100\) [\textdegree{}C], \(T_w(0) = 70\)[\textdegree{}C], \(m_w = 2.5\)[kg], \(t = 2\)[s]}
\end{figure}

\begin{figure}[H]
	\includegraphics[width=\linewidth]{../assets/part2/zmienny-h-7.png}
	\caption{Parametry: \(T_b(0) = 1100\) [\textdegree{}C], \(T_w(0) = 70\)[\textdegree{}C], \(m_w = 5\)[kg], \(t = 2\)[s]}

	\includegraphics[width=\linewidth]{../assets/part2/zmienny-h-8.png}
	\caption{Parametry: \(T_b(0) = 1100\) [\textdegree{}C], \(T_w(0) = 70\)[\textdegree{}C], \(m_w = 10\)[kg], \(t = 2\)[s]}
\end{figure}

\begin{figure}[H]
	\includegraphics[width=\linewidth]{../assets/part2/zmienny-h-9.png}
	\caption{Parametry: \(T_b(0) = 1100\) [\textdegree{}C], \(T_w(0) = 70\)[\textdegree{}C], \(m_w = 2.5\)[kg], \(t = 4\)[s]}

	\includegraphics[width=\linewidth]{../assets/part2/zmienny-h-10.png}
	\caption{Parametry: \(T_b(0) = 1100\) [\textdegree{}C], \(T_w(0) = 70\)[\textdegree{}C], \(m_w = 2.5\)[kg], \(t = 5\)[s]}
\end{figure}

\begin{figure}[H]
	\includegraphics[width=\linewidth]{../assets/part2/zmienny-h-11.png}
	\caption{Parametry: \(T_b(0) = 1100\) [\textdegree{}C], \(T_w(0) = 70\)[\textdegree{}C], \(m_w = 2.5\)[kg], \(t = 4\)[s]}

	\includegraphics[width=\linewidth]{../assets/part2/zmienny-h-12.png}
	\caption{Parametry: \(T_b(0) = 1100\) [\textdegree{}C], \(T_w(0) = 70\)[\textdegree{}C], \(m_w = 2.5\)[kg], \(t = 5\)[s]}
\end{figure}

\begin{table}[H]
	\begin{tabular}{|c|c|c|c|}\hline%
	Nr próby & oczekiwane \(T_b(t)\)[\textdegree{}C] & przybliżone \(T_b(t)\) [\textdegree{}C] & \(\delta T_b(t)\) \\\hline
	\csvreader[no head, late after line=\\\hline]%
	{../assets/part2/wyniki.csv}{}%
	{\thecsvrow&\csvcoli&\csvcoliii&\csvcoliv}%
	\end{tabular}
	\caption{Temperatury końcowe pręta przybliżone za pomocą ulepszonej metody Eulera z aproksymowanym współczynnikiem przewodnictwa ciepła \(h\)}
	\label{tab:approx-wyniki-1}
\end{table}

\begin{table}[H]
	\begin{tabular}{|c|c|c|c|}\hline%
	Nr próby & oczekiwane \(T_w(t)\)[\textdegree{}C] & przybliżone \(T_w(t)\) [\textdegree{}C] & \(\delta T_w(t)\) \\\hline
	\csvreader[no head, late after line=\\\hline]%
	{../assets/part2/wyniki.csv}{}%
	{\thecsvrow&\csvcolii&\csvcolv&\csvcolvi}%
	\end{tabular}
	\caption{Temperatury końcowe oleju chłodzącego przybliżone za pomocą ulepszonej metody Eulera z aproksymowanym współczynnikiem przewodnictwa ciepła \(h\)}
	\label{tab:approx-wyniki-2}
\end{table}

W celu zbadania wpływu modelu aproksymowanego charakterystykę współczynnika przewodnictwa cieplnego \(h\) na wyniki symulacji
temperatury pręta oraz oleju chłodzącego zostały powtórzone próby eksperymentalne z \textbf{części I}
z użyciem ulepszonej metody Eulera.
Dodatkowo został obliczony błąd temperatury końcowej pręta oraz oleju chłodzącego po symulacji
względem danych pozyskanych w eksperymencie fizycznym.

\subsection{Wnioski}
Aproksymowana charakterystyka współczynnika przewodnictwa cieplnego \(h\) ma bardzo niewielki wpływ
na wyniki symulacji temperatury pręta oraz oleju chłodzącego.

\newpage

\section{Część 3}
\subsection{Wstęp}
Część trzecia projektu polega na wyznaczeniu minimalnej wartości masy oleju chłodzącego niezbędnego
do wychłodzenia pręta w pewnym czasie \(t\) do określonej temperatury. \\

\textbf{Metoda Newtona-Raphsona} została wykorzystana do rozwiązania tego zagadnienia. \\
W przypadku tego zagadnienia przyjmuje postać:
\begin{flalign*}
	m^{(i+1)}_w = m^{(i)}_w - \frac{T_b(m^{(i)}_w)}{T_b'(m^{(i)}_w)}
\end{flalign*}
Aby wyznaczyć wartość pochodnej należy użyć jej numerycznego przybliżenia:
\begin{flalign*}
	T_b'(m_w) = \frac{dT_b}{dm_w} |_{t=0.7s} \approx \frac{T_b(m_w + \Delta m_w) - T_b(m_w)}{\Delta m_w}
\end{flalign*}
gdzie: \\
\(\Delta m_w\) jest małą różnicą masy płynu chłodzącego.

\subsubsection{Metoda Newtona-Raphsona}
Zakładamy, że funkcja \(f\) jest klasy \(C^2(a,b)\),
zmienia znak w przedziale \((a,b)\) oraz pochodne pierwsza i druga mają stały znak w rozpatrywanym przedziale.
\\
Jako punkt startu obieramy taki punkt \(x_0\), w którym funkcja ma identyczny znak jak druga pochodna:
\begin{align*}
	f(x_0)f''(x_0) > 0
\end{align*}
Z punktu \((x_0, f(x_0))\) wystawiamy styczną do krzywej \(y = f(x)\):
\begin{align*}
	y - f(x_0) = f'(x_0)(x-x_0)
\end{align*}
Następnie przecinamy tą styczną z osią \(0x\) i otrzymany w ten sposób punkt przecięcia
jest pierwszym przybliżeniem pierwiastka.


\subsection{Podsumowanie współczynników}
 \begin{flalign*}
 	& h = h(\Delta T) \quad [\frac{J}{s \cdot m^2}] \\
 	& A = 0.0109 \quad [m^2] \\
 	& m_b = 0.25 \quad [kg] \\
 	& m_w = ? \quad [kg] \\
 	& c_b = 0.29 \quad [\frac{J}{kg \cdot K}] \\
 	& c_w = 4.1813 \quad [\frac{J}{kg \cdot K}] \\
 	& t = 0.7 \quad [s]
 \end{flalign*}
\subsection{Kod źródłowy programu}
\lstinputlisting[language=MatLab, caption=Funkcja sterująca z danymi parametrami oraz rysująca wykresy]{../src/part3.m}
\subsection{Interpretacja wyników}
\begin{figure}[H]
	\includegraphics[width=\linewidth]{../assets/part3/temperatura-preta-1.png}
	\caption{Parametry: \(T_b(0) = 1200\) [\textdegree{}C], \(T_w(0) = 25\)[\textdegree{}C], \(m_w = 0.05\)[kg], \(t = 0.7\)[s]}

	\includegraphics[width=\linewidth]{../assets/part3/temperatura-preta-2.png}
	\caption{Parametry: \(T_b(0) = 1200\) [\textdegree{}C], \(T_w(0) = 25\)[\textdegree{}C], \(m_w = 0.1\)[kg], \(t = 0.7\)[s]}
\end{figure}

\begin{figure}[H]
	\includegraphics[width=\linewidth]{../assets/part3/temperatura-preta-2.png}
	\caption{Parametry: \(T_b(0) = 1200\) [\textdegree{}C], \(T_w(0) = 25\)[\textdegree{}C], \(m_w = 0.15\)[kg], \(t = 0.7\)[s]}

	\includegraphics[width=\linewidth]{../assets/part3/temperatura-preta-4.png}
	\caption{Parametry: \(T_b(0) = 1200\) [\textdegree{}C], \(T_w(0) = 25\)[\textdegree{}C], \(m_w = 0.2\)[kg], \(t = 0.7\)[s]}
\end{figure}

\subsubsection{Wnioski}
Przebieg czasowy temperatury pręta maleje proporcjonalnie do wzrostu masy oleju chłodzącego \(m_w\).

\section{Część 4}
\subsection{Wstęp}
Ostatnia część projektu polega na zaprojektowaniu i implementacji złożonego programu,
pozwalającego określić optymalne parametry zbiornika z olejem chłodzącym
pod względem minimalizacji kosztów produkcji, biorąc pod uwagę pewne narzucone ograniczenia.

\subsubsection{Metoda bisekcji}
Zakładamy, że funkcja \(f\) jest ciągła na przedziale \([a,b]\) oraz:
\begin{flalign*}
	f(a)f(b) < 0
\end{flalign*}
Dzielimy przedział \([a,b]\) na dwie połowy w punkcie:
\begin{flalign*}
	x_1 = \frac{a + b}{2}
\end{flalign*}
Następnie:
\begin{enumerate}
	\item Wybieramy taki przedział, w którym funkcja \(f\) zmienia znak
	\item Następnie dzielimy go na połowę punktem \(x_2\)
	\item Czynność powtarzamy do momentu osiągnięcia ustalonego wcześniej warunku
\end{enumerate}

Jeżeli \(f(x_1) = 0\), wówczas \(x_1\) jest pierwiastkiem równania. \\
Nastomiast jeżeli \(f(x_1)\) jest różne od zera to z otrzymanych dwóch przedziałów:
\([a, x_1]\) i \([x_1, b]\) wybieramy ten, w którym funkcja \(f\) zmienia znak. \\
Następnie przedział ten dzielimy na połowy punktem \(x_2\) i badamy wartość funkcji w tym punkcie.
Powtarzamy tą czynność \(n\) razy. \\
Otrzymujemy \(f(x_n) = 0\) lub ciąg podprzedziałów takich, że:
\begin{flalign*}
	f(x_n)f(x_{n+1}) < 0
\end{flalign*}
gdzie \(x_n, x_{n+1}\) są końcami przedziału, a jego długość wynosi:
\begin{flalign*}
	|x_n - x_{n+1}| < \frac{1}{2^n}(b-a)
\end{flalign*}
Tereotycznie można uzyskać dowolną dokładność przy obliczeniach pierwiastka, stosujac iteracje
do spełnienia warunku:
\begin{flalign*}
	|x_n - x_{n+1}| < \frac{1}{2^n}(b-a) < \epsilon
\end{flalign*}
\subsection{Podsumowanie współczynników}
\begin{itemize}
	\item W ciągu 24 godzin należy schłodzić minimum 100 000 prętów.
	\item Wymiana jednego pręta wynosi 5 sekund.
	\item Posiadamy maksymalnie 100 stanowisk chłodzenia.
	\item Posiadamy nieskończoną liczbę zbiorników chłodzących
	\item Czas wymiany jednego zbiornika wynosi 30 sekund.
	\item Czas schłodzenia zbiornika z temperatury 125 stopni do 25 stopni zależy
		liniowo od masy płynu chłodzącego:
		\begin{flalign*}
			t = m_w \cdot 0.15 \quad [h]
		\end{flalign*}
	\item Koszt wyprodukowania zbiornika zależy od masy:
	\begin{flalign*}
		k_{zb} = 100 \cdot m_w \quad [PLN]
	\end{flalign*}
	\item Koszt oleju chłodzącego wynosi:
	\begin{flalign*}
		k_w = 20 \cdot m_w \quad [PLN]
	\end{flalign*}
	\item Zakładamy, że po schłodzeniu 2000 prętów olej chłodzący musi być wymieniony.
	\item Temperatura początkowa pręta wynosi 1200 \textdegree{}C
	\item Temperatura początkowa oleju chłodzącego wynosi 25 \textdegree{}C
	\item Temperatura końcowa pręta po schłodzeniu wynosi 125 \textdegree{}C
	\item Dodatkowym założeniem jest, że zbiornik należy zacząć chłodzić,
	kiedy olej chłodzący osiągnął temperaturę większą niż 100 \textdegree{}C

\end{itemize}

\newpage

\subsection{Kod źródłowy programu}
\lstinputlisting[language=MatLab, caption=Funkcja sterująca z danymi parametrami oraz rysująca wykresy]{../src/part4.m}
\lstinputlisting[language=MatLab, caption=Funkcja obliczająca minimalną masę \(m_w\) metodą Newtona-Raphsona]{../src/getMinMw.m}
\lstinputlisting[language=MatLab, caption=Funkcja obliczająca czas chłodzenia zbiornika]{../src/getCoolingTime.m}
\lstinputlisting[language=MatLab, caption=Funkcja obliczająca ilość potrzebnych zbiorników{../src/getContainerNum.m}
\lstinputlisting[language=MatLab, caption=Funkcja obliczająca koszt oleju chłodzącego]{../src/getOilCost.m}
\lstinputlisting[language=MatLab, caption=Funkcja obliczająca koszt zbiornika]{../src/getContainerCost.m}
\lstinputlisting[language=MatLab, caption=Funkcja obliczająca koszt całościowy]{../src/getCost.m}
\subsection{Interpretacja wyników}

Ze względu na to, że wszystkie stacje chłodzące pracują w tym samym czasie z identyczną liczbą prętów przypadającą
na stacje obliczenia zostały przeprowadzone na przykładzie jednej stacji chłodzącej, a całkowity koszt chłodzenia przypadajacej
liczby prętów został przemnożony przez ilość stacji.

\begin{figure}[H]
	\includegraphics[width=\linewidth]{../assets/part4/ilosc-zbiornikow-55.png}
	\caption{Parametry: Ilość stacji chłodzących: 55}

	\includegraphics[width=\linewidth]{../assets/part4/ilosc-zbiornikow-78.png}
	\caption{Parametry: Ilość stacji chłodzących: 78}
\end{figure}


\subsection{Wnioski}

\newpage

\listoftables
\lstlistoflistings
\listoffigures

\end{document}
