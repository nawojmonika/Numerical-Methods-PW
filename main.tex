%%%%%%%%%%%%%%%%%%%%%%%%%%%%%%%%%%%%%%%%%
% fphw Assignment
% LaTeX Template
% Version 1.0 (27/04/2019)
%
% This template originates from:
% https://www.LaTeXTemplates.com
%
% Authors:
% Class by Felipe Portales-Oliva (f.portales.oliva@gmail.com) with template
% content and modifications by Vel (vel@LaTeXTemplates.com)
%
% Template (this file) License:
% CC BY-NC-SA 3.0 (http://creativecommons.org/licenses/by-nc-sa/3.0/)
%
%%%%%%%%%%%%%%%%%%%%%%%%%%%%%%%%%%%%%%%%%

%----------------------------------------------------------------------------------------
%	PACKAGES AND OTHER DOCUMENT CONFIGURATIONS
%----------------------------------------------------------------------------------------

\documentclass[
	12pt, % Default font size, values between 10pt-12pt are allowed
	%letterpaper, % Uncomment for US letter paper size
	%spanish, % Uncomment for Spanish
]{fphw}

% Template-specific packages
\usepackage[utf8]{inputenc} % Required for inputting international characters
\usepackage[T1]{fontenc} % Output font encoding for international characters
\usepackage{mathpazo} % Use the Palatino font

\usepackage{graphicx} % Required for including images

\usepackage{booktabs} % Required for better horizontal rules in tables

\usepackage{listings} % Required for insertion of code

\usepackage{enumerate} % To modify the enumerate environment

%----------------------------------------------------------------------------------------
%	ASSIGNMENT INFORMATION
%----------------------------------------------------------------------------------------

\title{Symulator chłodzenia pręta w oleju chłodzącym} % Assignment title

\author{Monika Nawój} % Student name

\date{14.11.2020} % Due date

\institute{Politechnika Warszawska \\ Wydział Elektryczny} % Institute or school name

\class{Metody Numeryczne 2020Z} % Course or class name

\professor{Rober Szmurło} % Professor or teacher in charge of the assignment

%----------------------------------------------------------------------------------------

\begin{document}

\maketitle % Output the assignment title, created automatically using the information in the custom commands above

%----------------------------------------------------------------------------------------
%	ASSIGNMENT CONTENT
%----------------------------------------------------------------------------------------

\section{Opis zjawiska}
\section{Część 1}
\subsection{Podsumowanie współczynników}
\subsection{Kod źródłowy progamu}
\subsection{Interpretacja wyników}
\subsection{Analiza rozwiązania}
\section{Część 2}
\subsection{Podsumowanie współczynników}
\subsection{Kod źródłowy progamu}
\subsection{Interpretacja wyników}
\subsection{Analiza rozwiązania}
\section{Część 3}
\subsection{Podsumowanie współczynników}
\subsection{Kod źródłowy progamu}
\subsection{Interpretacja wyników}
\subsection{Analiza rozwiązania}
\section{Część 4}
\subsection{Podsumowanie współczynników}
\subsection{Kod źródłowy progamu}
\subsection{Interpretacja wyników}
\subsection{Analiza rozwiązania}
\section{Podsumowanie}
\end{document}
